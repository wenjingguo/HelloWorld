

\documentclass[twoside]{article}



\usepackage[sc]{mathpazo} % Use the Palatino font
\usepackage[T1]{fontenc} % Use 8-bit encoding that has 256 glyphs
\linespread{1.25} % Line spacing - Palatino needs more space between lines
\usepackage{microtype} % Slightly tweak font spacing for aesthetics
\usepackage{mathptmx}


\usepackage[hmarginratio=1:1,top=20mm,bottom = 10 mm ,columnsep=20pt]{geometry} % Document margins
\usepackage{multicol} % Used for the two-column layout of the document
\usepackage[hang, small,labelfont=bf,up,textfont=it,up]{caption} % Custom captions under/above floats in tables or figures
\usepackage{booktabs} % Horizontal rules in tables
\usepackage{float} % Required for tables and figures in the multi-column environment - they need to be placed in specific locations with the [H] (e.g. \begin{table}[H])
\usepackage{hyperref} % For hyperlinks in the PDF

\usepackage{lettrine} % The lettrine is the first enlarged letter at the beginning of the text
\usepackage{paralist} % Used for the compactitem environment which makes bullet points with less space between them

\usepackage{abstract} % Allows abstract customization
\renewcommand{\abstractnamefont}{\normalfont\bfseries} % Set the "Abstract" text to bold
\renewcommand{\abstracttextfont}{\normalfont\small\itshape} % Set the abstract itself to small italic text




%----------------------------------------------------------------------------------------
%	TITLE SECTION
%----------------------------------------------------------------------------------------

\title{\vspace{-15mm}\fontsize{24pt}{10pt}\selectfont\textbf{Monte Carlo simulation strategies to compute bulk properties of octane-water mixture}} % Article title

\date{\today}


%----------------------------------------------------------------------------------------

\begin{document}

\maketitle % Insert title


%----------------------------------------------------------------------------------------
%	ABSTRACT
%----------------------------------------------------------------------------------------

\begin{abstract}



\end{abstract}

%----------------------------------------------------------------------------------------
%	ARTICLE CONTENTS
%----------------------------------------------------------------------------------------

\begin{multicols}{2} % Two-column layout throughout the main article text

\section{Introduction}




%------------------------------------------------

\section{Methods}

 



%------------------------------------------------

\section{Results and Discussion}
To predict the bulk phase behavior of the mixture, the influence parameter of the pure components must be known. Therefore, first bulk properties of pure component are discussed, and then the bulk properties of the mixture are investigated.
\subsection{Pure component}
In this section,we will present our results for pure water and octane respectively. The bulk phase behavior of both the model fluids has been studied by our group and several other authors. However, the non-truncated Lennard-Jones model was considered in these earlier studies. Due to the difficulties involved in  modeling long-range interactions within inhomogeneous systems, we opted to wrok with a truncated fluid-fluid interactions here. Therefore, we are not able to directly compare our data to those from previous studies. In other words, the data we obtained from the current simulation are qualitatively consistent with the results from previous studies.
\subsubsection{Pure water}

\subsection{Octane-water mixture}
\subsubsection{Direct calculation}
\subsubsection{Activity fraction expanded ensemble}
\subsubsection{Temperature expanded ensemble}


\section{Conclusion}

%------------------------------------------------

%----------------------------------------------------------------------------------------
%	REFERENCE LIST
%----------------------------------------------------------------------------------------

\begin{thebibliography}{99} % Bibliography - this is intentionally simple in this template


 
\end{thebibliography}

%----------------------------------------------------------------------------------------

\end{multicols}

\end{document}
